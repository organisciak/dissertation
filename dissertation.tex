\documentclass[nobib,oneside,openany]{tufte-book}

%\usepackage[titles]{tocloft}
%\renewcommand{\cftpartleader}{\cftdotfill{\cftdotsep}} % for parts
%\renewcommand{\cftchapleader}{\cftdotfill{\cftdotsep}} % for chapters

\hypersetup{colorlinks}% uncomment this line if you prefer colored hyperlinks (e.g., for onscreen viewing)

%%
% Book metadata

\title{Crowdsourcing metadata: Reliably augmenting documents by collecting and interpreting amateur contributions}
\author{Peter Organisciak}
\publisher{Graduate School of Library and Information Science}

%%
% If they're installed, use Bergamo and Chantilly from www.fontsite.com.
% They're clones of Bembo and Gill Sans, respectively.
%\IfFileExists{bergamo.sty}{\usepackage[osf]{bergamo}}{}% Bembo
%\IfFileExists{chantill.sty}{\usepackage{chantill}}{}% Gill Sans

%\usepackage{microtype}

% Don't remember why I had this, so commenting out until I find something broken!
%\usepackage[utf8]{inputenc}

% Tables created by Pandoc
\usepackage{longtable}

% Biblatex via biber
\usepackage[english]{babel}% Recommended
\usepackage[babel]{csquotes} % Recommended
\usepackage[backend=biber,bibencoding=utf8,style=authoryear]{biblatex}
\bibliography{refs.bib}

%%

%%
\usepackage{tabularx}
% For nicely typeset tabular material
\usepackage{booktabs}
\usepackage{setspace}

%% For \text in match equations
\usepackage{amsmath}


%%
% For graphics / images
\usepackage{graphicx}
\setkeys{Gin}{width=\linewidth,totalheight=\textheight,keepaspectratio}
\graphicspath{{graphics/}}

% The fancyvrb package lets us customize the formatting of verbatim
% environments.  We use a slightly smaller font.
\usepackage{fancyvrb}
\fvset{fontsize=\normalsize}

%% CUSTOM ADDED
%% PANDOC-specific
\providecommand{\tightlist}{%
	  \setlength{\itemsep}{0pt}\setlength{\parskip}{0pt}}

%%%%%%

 %% TUFTE COMMANDS
% Prints argument within hanging parentheses (i.e., parentheses that take
% up no horizontal space).  Useful in tabular environments.
\newcommand{\hangp}[1]{\makebox[0pt][r]{(}#1\makebox[0pt][l]{)}}

%%
% Prints an asterisk that takes up no horizontal space.
% Useful in tabular environments.
\newcommand{\hangstar}{\makebox[0pt][l]{*}}

%%
% Prints a trailing space in a smart way.
\usepackage{xspace}

%%



% Prints the month name (e.g., January) and the year (e.g., 2008)
\newcommand{\monthyear}{%
  \ifcase\month\or January\or February\or March\or April\or May\or June\or
  July\or August\or September\or October\or November\or
  December\fi\space\number\year
}


% Prints an epigraph and speaker in sans serif, all-caps type.
\newcommand{\openepigraph}[2]{%
  %\sffamily\fontsize{14}{16}\selectfont
  \begin{fullwidth}
  \sffamily\large
  \begin{doublespace}
  \noindent\allcaps{#1}\\% epigraph
  \noindent\allcaps{#2}% author
  \end{doublespace}
  \end{fullwidth}
}

% Inserts a blank page
\newcommand{\blankpage}{\newpage\hbox{}\thispagestyle{empty}\newpage}

\usepackage{units}

% Typesets the font size, leading, and measure in the form of 10/12x26 pc.
\newcommand{\measure}[3]{#1/#2$\times$\unit[#3]{pc}}

% Macros for typesetting the documentation
\newcommand{\hlred}[1]{\textcolor{Maroon}{#1}}% prints in red
\newcommand{\hangleft}[1]{\makebox[0pt][r]{#1}}
\newcommand{\hairsp}{\hspace{1pt}}% hair space
\newcommand{\hquad}{\hskip0.5em\relax}% half quad space
\newcommand{\TODO}{\textcolor{red}{\bf TODO!}\xspace}
\newcommand{\ie}{\textit{i.\hairsp{}e.}\xspace}
\newcommand{\eg}{\textit{e.\hairsp{}g.}\xspace}
\newcommand{\na}{\quad--}% used in tables for N/A cells
\providecommand{\XeLaTeX}{X\lower.5ex\hbox{\kern-0.15em\reflectbox{E}}\kern-0.1em\LaTeX}
\newcommand{\tXeLaTeX}{\XeLaTeX\index{XeLaTeX@\protect\XeLaTeX}}
% \index{\texttt{\textbackslash xyz}@\hangleft{\texttt{\textbackslash}}\texttt{xyz}}
\newcommand{\tuftebs}{\symbol{'134}}% a backslash in tt type in OT1/T1
\newcommand{\doccmdnoindex}[2][]{\texttt{\tuftebs#2}}% command name -- adds backslash automatically (and doesn't add cmd to the index)
\newcommand{\doccmddef}[2][]{%
  \hlred{\texttt{\tuftebs#2}}\label{cmd:#2}%
  \ifthenelse{\isempty{#1}}%
    {% add the command to the index
      \index{#2 command@\protect\hangleft{\texttt{\tuftebs}}\texttt{#2}}% command name
    }%
    {% add the command and package to the index
      \index{#2 command@\protect\hangleft{\texttt{\tuftebs}}\texttt{#2} (\texttt{#1} package)}% command name
      \index{#1 package@\texttt{#1} package}\index{packages!#1@\texttt{#1}}% package name
    }%
}% command name -- adds backslash automatically
\newcommand{\doccmd}[2][]{%
  \texttt{\tuftebs#2}%
  \ifthenelse{\isempty{#1}}%
    {% add the command to the index
      \index{#2 command@\protect\hangleft{\texttt{\tuftebs}}\texttt{#2}}% command name
    }%
    {% add the command and package to the index
      \index{#2 command@\protect\hangleft{\texttt{\tuftebs}}\texttt{#2} (\texttt{#1} package)}% command name
      \index{#1 package@\texttt{#1} package}\index{packages!#1@\texttt{#1}}% package name
    }%
}% command name -- adds backslash automatically
\newcommand{\docopt}[1]{\ensuremath{\langle}\textrm{\textit{#1}}\ensuremath{\rangle}}% optional command argument
\newcommand{\docarg}[1]{\textrm{\textit{#1}}}% (required) command argument
\newenvironment{docspec}{\begin{quotation}\ttfamily\parskip0pt\parindent0pt\ignorespaces}{\end{quotation}}% command specification environment
\newcommand{\docenv}[1]{\texttt{#1}\index{#1 environment@\texttt{#1} environment}\index{environments!#1@\texttt{#1}}}% environment name
\newcommand{\docenvdef}[1]{\hlred{\texttt{#1}}\label{env:#1}\index{#1 environment@\texttt{#1} environment}\index{environments!#1@\texttt{#1}}}% environment name
\newcommand{\docpkg}[1]{\texttt{#1}\index{#1 package@\texttt{#1} package}\index{packages!#1@\texttt{#1}}}% package name
\newcommand{\doccls}[1]{\texttt{#1}}% document class name
\newcommand{\docclsopt}[1]{\texttt{#1}\index{#1 class option@\texttt{#1} class option}\index{class options!#1@\texttt{#1}}}% document class option name
\newcommand{\docclsoptdef}[1]{\hlred{\texttt{#1}}\label{clsopt:#1}\index{#1 class option@\texttt{#1} class option}\index{class options!#1@\texttt{#1}}}% document class option name defined
\newcommand{\docmsg}[2]{\bigskip\begin{fullwidth}\noindent\ttfamily#1\end{fullwidth}\medskip\par\noindent#2}
\newcommand{\docfilehook}[2]{\texttt{#1}\index{file hooks!#2}\index{#1@\texttt{#1}}}
\newcommand{\doccounter}[1]{\texttt{#1}\index{#1 counter@\texttt{#1} counter}}

% Generates the index
\usepackage{makeidx}
\makeindex


%%% PO: Additional customizations in tufte-book-local.sty
% Float longtable captions (the tables used by Pandoc)
% to the margin, as is done for figures
\makeatletter
\def\LT@makecaption#1#2#3{%
  \noalign{\smash{\hbox{\kern\textwidth\rlap{\kern\marginparsep
  \parbox[t][][b]{\marginparwidth}{%
  \vphantom{40pt} % PO: hacking around a slight verticle misplacement
\@tufte@caption@font \@tufte@caption@justification \noindent 
   #1{#2: }\ignorespaces #3}}}}}}
\makeatother


% Set a hard margin of one inch.
% from http://tex.stackexchange.com/questions/119900/how-do-i-set-a-hard-minimum-margin-size-with-the-tufte-book-document-class
% For US letter paper
\geometry{
  left=1in, % left margin
  textwidth=25pc, % main text block
  marginparsep=2pc, % gutter between main text block and margin notes
  marginparwidth=12pc % width of margin notes
}




% Work on proper arbaic vs roman numerals

\renewcommand\frontmatter{%
  \pagenumbering{roman}%
}

\renewcommand\mainmatter{%
  \pagenumbering{arabic}%
}


%

\begin{document}
% Front matter
\frontmatter

% r.3 full title page
\maketitle

% PO Abstract needs to precede everything but the title page, starting
% roman numeral 2
% For print, it makes sense to use cleardoublepage to skip a page, but 
% for deposit, they want it on page 2
%\cleardoublepage
\setcounter{page}{2}
\pagestyle{plain}
\pagenumbering{roman}
\include{0_abstract}

% v.4 epigraphs and copyright page
\newpage

%\openepigraph{%
%In these democratic days, any investigation in the trustworthiness and peculiarities of popular judgments is of interest
%}{Francis Galton, {\itshape Vox Populi, 1907}
%}



%\begin{fullwidth}
%~\vfill
%\thispagestyle{empty}
%\setlength{\parindent}{0pt}
%\setlength{\parskip}{\baselineskip}
%Copyright \copyright\ \the\year\ \thanklessauthor

%\par\textit{DRAFT, \monthyear}
%\end{fullwidth}

% r.5 contents
\tableofcontents

% PO: forcing table and figure lists to the TOC, 
% but it may just be easier to remove these options and
% not particularly recommended sections for deposit

%\cleardoublepage
%\addcontentsline{toc}{chapter}{\listfigurename}
%\listoffigures

%\cleardoublepage
%\addcontentsline{toc}{chapter}{\listtablename}
%\listoftables

% r.7 dedication
%\cleardoublepage
%~\vfill
%\begin{doublespace}
%\noindent\fontsize{18}{22}\selectfont\itshape
%\nohyphenation
%Dedicated to those who appreciate \LaTeX{}
%and the work of \mbox{Edward R.~Tufte}
%and \mbox{Donald E.~Knuth}.
%\end{doublespace}
%\vfill
%\vfill

\cleardoublepage

% Start the main matter (normal chapters)
\mainmatter

% r.9 introduction
% PO: for deposit, abstract needs to be on the next page, but for print,
% it will make sense to use cleardoublepage (empty page in between)
%\cleardoublepage
\onehalfspacing
% PO set page numbering to start from 1
\setcounter{page}{1}
\chapter{INTRODUCTION}\label{introduction}

\begin{quote}
In these democratic days, any investigation in the trustworthiness and
peculiarities of popular judgments is of interest -- Francis Gaston,
1907
\end{quote}

\section{Notes}\label{notes}

\section{Introduction}\label{introduction-1}

Strong information retrieval depends on reliable, detailed information
to index. The broad phenomenon of crowdsourcing has the potential to
improve retrieval over web documents by producing descriptive metadata
about documents. Since crowdsourcing considers humans at large-scales,
it can be used for qualitative and subjective information at scales
useful to retrieval.

\subsection{Broad Research Question}\label{broad-research-question}

However, humans have predictable and unpredictable biases that make it
difficult to systematically adopt their contributions in an information
system. How do we interpret qualitative user contributions in an
inherently quantitative system? This study looks at the effect of human
biases on crowdsourcing in information retrieval and how they affects
the product of human contributions.

\subsection{Specific Research
Question}\label{specific-research-question}

Specifically: can bias in \emph{descriptive crowdsourcing} be accounted
for in a manner that improves the information-theoretic quality of the
contribution, either at the time of data collection or afterward?

\subsubsection{Hypothesis}\label{hypothesis}

The proposed study makes an assumption that crowd contributors are
honest but inherently biased, with the hypothesis that such a assumption
leads to a) more algorithmically valuable crowdsourced description and
b) a greater proportion of useful contributions.

\subsubsection{Approach}\label{approach}

This hypothesis will be applied in two different sites of crowdsourcing:
in the design of contribution tasks in order to minimize bias, and in
the normalization of contributions after they have already been
collected. Doing so will both adopt work that I have performed during my
doctoral studies and perform new research.

In looking at the design of contribution tasks, I hope to concentrate on
paid crowdsourcing. For the scope of this study, it would be intractable
to look at at the design of both paid and volunteer crowdsourcing, so I
will pursue the facet more pertinent to information retrieval. Paying
workers is only a subset of crowdsourcing approaches, and one that
arguably tethers the scalability of a task by anchoring it to financial
means. However, it is easier to control for by removing much of the
complexities of motivation. Information retrieval researchers are using
the predictability of paid crowd markets like Amazon's Mechanical Turk
to generate on-demand data, making design for those systems important.

More importantly, much information retrieval research occurs parallel to
the system of content information retrieval systems

\subsubsection{Practical application}\label{practical-application}

The contribution of this work is the application of human corrective
techniques to the encoding of metadata about existing information
object, and the broader understanding of the nature of such
contributions.

\subsection{Relevance to IS and IR}\label{relevance-to-is-and-ir}

Information science deals with many information objects, giving
crowdsourcing considerable potential as a tool for item description. By
collecting human judgments about the quality of information\ldots{}

\subsection{Posterior Corrections of
Bias}\label{posterior-corrections-of-bias}

\section{Chapter Outline}\label{chapter-outline}

The proposed dissertation will follow the following structure,
delineated by chapters.

\subsubsection{Introduction}\label{introduction-2}

The first chapter will introduce the problem of human bias in
crowdsourcing and how it affects computational uses of contributed data.
Subsequently, the assumption of honest but biased contributors will be
outlined, and the hypothesis on this assumption will be outlined along
with the study that will be pursued to test it.

\subsubsection{Literature Review}\label{literature-review}

\subsubsection{A Priori Corrections for
Bias}\label{a-priori-corrections-for-bias}

\subsubsection{Posterior Corrections for
Bias}\label{posterior-corrections-for-bias}
	% for INTRODUCTION in "intro.tex"

%%

\include{2_crowdsourcing}

\include{3_typology}

\include{4_posterior_objective}

\include{5_design_objective}

\include{5_5_ams-study}

\include{6_subjective}

\include{7_conclusions}

\backmatter

\include{appendix_contributions}

%\include{appendix_code}


%
% REMOVED BIBTEX
%\bibliography{sample-handout}
%\bibliographystyle{plainnat}


% ADDED BIBLATEX
\clearpage
% force bib to toc
\printbibliography[heading=bibintoc]

\printindex

\end{document}
