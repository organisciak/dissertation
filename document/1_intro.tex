\chapter{INTRODUCTION}\label{introduction}

\begin{quote}
In these democratic days, any investigation in the trustworthiness and
peculiarities of popular judgments is of interest -- Francis Gaston,
1907
\end{quote}

\% TODO: Add citation to refs

\section{Notes}\label{notes}

\section{Introduction}\label{introduction-1}

Strong information retrieval depends on reliable, detailed information
to index. The broad phenomenon of crowdsourcing has the potential to
improve retrieval over web documents by producing descriptive metadata
about documents.

\subsection{Broad Research Question}\label{broad-research-question}

How do we interpret these qualitative user contributions in a
quantitative system? Humans have individual biases that results in
differences between how a contribution is created: this study looks at
how such variance is introduced and how it affects the product of human
contributions.

\section{Specific Research Question}\label{specific-research-question}

Specifically, this study asks whether bias in /descriptive
crowdsourcing/ be accounted for in a manner that improves the
information-theoretic quality of the contribution, either at the time of
data collection or afterward.

\subsubsection{Hypothesis}\label{hypothesis}

The proposed study makes an assumption that crowd contributors are
honest but inherently biased, with the hypothesis that such a assumption
can result in a) more algorithmically valuable crowdsourced discription
and b) a greater proportion of useful contributions.

\subsubsection{Practical application}\label{practical-application}

The contribution of this work is the application of human corrective
techniques to the encoding of metadata about existing information
object, and the broader understanding of the nature of such
contributions.

TODO: crowdsourcing for encoding existing information with more
informative metadata, with a goal of improving information retrieval
systems. By focusing on a mix

\subsection{Relevance to IS and IR}\label{relevance-to-is-and-ir}

Information science deals with many information objects, giving
crowdsourcing considerable potential as a tool for item description. By
collecting human judgments about the quality of information\ldots{}

TODO

\subsubsection{Notes}\label{notes-1}

Discuss relevance of problem to information science and specifically
information retrieval.

This is partially strategic for myself: I should think of the problem
right out of the gate: carefully and precisely. Relevance to my areas of
study should be returned to in-depth later.

\section{Problem}\label{problem}

\subsection{Relevance in Information
Science}\label{relevance-in-information-science}

\subsection{Notes}\label{notes-2}

\begin{itemize}
\item
  Much crowdsourcing research makes an adversarial assumption
\item
  Crowdsourcing aggregates contributions from human
  participants/workers. While such contributions are helpful for
  understanding the content in an information system, they are
\end{itemize}

\subsection{Text}\label{text}

\section{Methodology}\label{methodology}

\subsection{Definitions}\label{definitions}

Before proceeding, the terminology of this study should be established.
As this work spans multiple domains, and makes reference to recently
introduced concepts, it is important to establish a shared understanding
of language within these pages.

Note that the treatment here is cursory; a more in-depth look can be
found in Chapter 2. (TODO better reference)

\subsubsection{Crowdsourcing}\label{crowdsourcing}

Crowdsou

\subsubsection{Descriptive
crowdsourcing}\label{descriptive-crowdsourcing}

\subsubsection{Human computation}\label{human-computation}

\subsubsection{Worker (paid)}\label{worker-paid}

\subsubsection{Volunteer, contributor}\label{volunteer-contributor}

\subsubsection{Human bias?}\label{human-bias}

\subsection{A Priori Corrections of
Bias}\label{a-priori-corrections-of-bias}

\subsubsection{Introduction}\label{introduction-2}

\subsubsection{Literature}\label{literature}

\subsubsection{My Research Thus Far}\label{my-research-thus-far}

\subsubsection{Proposed Research}\label{proposed-research}

In new research for this chapter, I will investigate the effect of
different parameterizations of the task,

\paragraph{Data}\label{data}

\paragraph{Parameterization}\label{parameterization}

\paragraph{Evaluation}\label{evaluation}

\subsection{Posterior Corrections of
Bias}\label{posterior-corrections-of-bias}

\section{Chapter Outline}\label{chapter-outline}

The proposed dissertation will follow the following structure,
delineated by chapters.

\subsubsection{Introduction}\label{introduction-3}

The first chapter will introduce the problem of human bias in
crowdsourcing and how it affects computational uses of contributed data.
Subsequently, the assumption of honest but biased contributors will be
outlined, and the hypothesis on this assumption will be outlined along
with the study that will be pursued to test it.

\subsubsection{Literature Review}\label{literature-review}

\subsubsection{A Priori Corrections for
Bias}\label{a-priori-corrections-for-bias}

\subsubsection{Posterior Corrections for
Bias}\label{posterior-corrections-for-bias}
